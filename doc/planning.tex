\documentclass{book}

\title{Planning an Economy}
\author{ShaderKitty}
\date{\today}

\begin{document}
\maketitle

\tableofcontents

\frontmatter
\chapter{Preface}

Production is inherently social in nature. The point of every mode of production throughout history has been to physically determine what work needs to be done, to produce a set of desired outputs. It is no different in a capitalist mode of production, and it will be no different in a socialist mode of production.
\chapter{Transformation of Labor}
The goal of every economy, is to determine the socially necessary labor time $\vec{l}$ to produce said set of commodities $\vec{q}$ of real goods. When it comes to the allocation of resources within the capitalist mode of production, the coordination of production is limited to only the information given through price by other firms. In a planned economy, with a corresponding set of firms $S$ are capable of sharing more information of their productive capabilities beyond just price signals.
\chapter{Independent Production}
In a simple economy, where the production of goods is limited to independent production, where the production of goods require no material inputs to produce said commodities, only pure labor contributes to the value of a commodity. It is the added value in production of that commodity.
\begin{equation}
	\label{Eq Direct Production}
	q_{i} p_{i} = l_{i} \quad i \in [1, n]
\end{equation}
Equation \ref{Eq Direct Production} is limited to only pure independent production of commodities. Price is however easily determined by dividing the total contributed labor by the total output quantity of production, yielding the man-hours per unit.
\chapter{Dependent Production}
\end{document}